
% The \phantomsection command is needed to create a link to a place in the document that is not a
% figure, equation, table, section, subsection, chapter, etc.
% https://tex.stackexchange.com/questions/44088/when-do-i-need-to-invoke-phantomsection
\phantomsection

% Multiple-language document - babel - selectlanguage vs begin/end{otherlanguage}
% https://tex.stackexchange.com/questions/36526/multiple-language-document-babel-selectlanguage-vs-begin-endotherlanguage
\begin{otherlanguage*}{brazil}

    \chapter{Cronograma}
    
    Até agora, o trabalho avançou pelas seções introdutórias, propondo e justificando a proposta da criação do plugin, e delineando claramente os objetivos, tanto gerais quanto específicos, da pesquisa. A metodologia utilizada no estudo foi apresentada, fornecendo uma estrutura para as fases subsequentes da pesquisa. A estrutura do trabalho foi detalhada, oferecendo um guia para o leitor. A revisão da literatura, abrangendo as bases teóricas, foi realizada, cobrindo tópicos relevantes como juízes online, plataformas como Beecrowd e ferramentas como Coderunner e Moodle. A seção de trabalhos relacionados explora pesquisas existentes, destacando estudos sobre Beecrowd, integração do BOCA com o Moodle e ferramentas como Coderunner para avaliação de habilidades de programação.

    Os trabalhos futuros incluiem testes da LTI do Beecrowd, a documentação desses testes, e a comunicação com os desenvolvedores da Beecrowd para resolver possíveis problemas. Além disso, será necessário ler e assistir tutoriais do Moodle sobre como criar um plugin e, posteriormente, desenvolver o Plugin Moodle para integrar Beecrowd com o Moodle. A fase de testes será crucial para avaliar o plugin desenvolvido, seguida pela documentação do processo de desenvolvimento e testes do plugin. Por fim, a finalização da monografia do Trabalho de Conclusão de Curso, juntamente com a redação do artigo sobre o TCC, encerrará o ciclo do projeto.

    As próximas etapas serão desenvolvidas ao longo do ano de 2024, visto que eu me matricularei na disciplina TCC 2 apenas no segundo semestre de 2024, a fim de desenvolver com calma as próximas etapas do projeto.

\begin{table}[htb]
  \IBGEtab{%
    \caption[Planejamento das etapas do trabalho de conclusão de curso]{Planejamento das etapas do trabalho de conclusão de curso}%
    \label{tabela-ibge}
  }{%
    \begin{tabular}{c c c c c c c c}
    \toprule
     \textbf{Etapas} & \textbf{Jan} & \textbf{Fev} & \textbf{Mar} & \textbf{Abr} & \textbf{Mai} & \textbf{Jun} & \textbf{Jul} \\
    \midrule
    Testes da LTI Beecrowd                 & X & X & X & X & X &   &          \\ \midrule
    Pesquisar como criar plugin            &   &   & X & X & X & X &          \\ \midrule 
    Desenvolver Plugin Moodle              &   &   & X & X & X & X & X        \\ \midrule
    Testar o plugin desenvolvido           &   &   &   &   & X & X & X        \\ \midrule
    Escrita do desenvolvimento do artigo   &   &   & X & X & X & X & X        \\ \midrule 
    Finalização da monografia do TCC       &   &   &   &   &   &   &         \\ 
    \bottomrule
  \end{tabular}%
  }{%
    \fonte{
        Produzido pela autora. 
    }%
  }
\end{table}

\begin{table}[htb]
  \IBGEtab{%
    \caption[Continuação do planejamento das etapas do trabalho de conclusão de curso]{Continuação do planejamento das etapas do trabalho de conclusão de curso}%
    \label{tabela-ibge}
  }{%
    \begin{tabular}{c c c c c}
    \toprule
     \textbf{Etapas} & \textbf{Ago} & \textbf{Set} & \textbf{Out} & \textbf{Nov} \\
    \midrule
    Testes da LTI Beecrowd                 &   &   &   &        \\ \midrule
    Pesquisar como criar plugin            &   &   &   &        \\ \midrule 
    Desenvolver Plugin Moodle              &   &   &   &        \\ \midrule
    Testar o plugin desenvolvido           &   &   &   &        \\ \midrule
    Escrita do desenvolvimento do artigo   & X &   &   &        \\ \midrule
    Finalização da monografia do TCC       & X & X & X & X      \\ 
    \bottomrule
  \end{tabular}%
  }{%
    \fonte{
        Produzido pela autora. 
    }%
  }
\end{table}

\end{otherlanguage*}

