
% The \phantomsection command is needed to create a link to a place in the document that is not a
% figure, equation, table, section, subsection, chapter, etc.
% https://tex.stackexchange.com/questions/44088/when-do-i-need-to-invoke-phantomsection
\phantomsection

% Multiple-language document - babel - selectlanguage vs begin/end{otherlanguage}
% https://tex.stackexchange.com/questions/36526/multiple-language-document-babel-selectlanguage-vs-begin-endotherlanguage
\begin{otherlanguage*}{brazil}

\chapter{Validação}

Nesta seção, será realizada uma análise crítica da proposta, abordando as vantagens e desvantagens da integração entre o Moodle e o Beecrowd via LTI (\textit{Learning Tools Interoperability}), bem como da aplicação da aplicação com sistema especialista desenvolvida. Além disso, será apresentada uma avaliação prática sobre o uso da aplicação pelos alunos da disciplina \textit{Programação Orientada a Objetos I} da Universidade Federal de Santa Catarina (UFSC) no segundo semestre de 2024, com o objetivo de verificar se o sistema especialista conseguiu resolver efetivamente as dúvidas dos estudantes.

\section{Integração LTI entre o Moodle e o Beecrowd}

\subsection{Vantagens}
\begin{itemize}
    \item \textbf{Vantagens do Beecrowd:} O Beecrowd oferece uma plataforma robusta para a resolução de problemas de programação, com uma grande variedade de questões que permitem aos alunos praticar e desenvolver habilidades de programação em diversos níveis de dificuldade. A plataforma também facilita o acompanhamento do progresso dos alunos, oferecendo métricas e \textit{feedback} instantâneo sobre o desempenho nas questões.
    \item \textbf{Facilidade de acesso através da integração LTI:} A integração LTI simplifica o acesso ao Beecrowd, pois tanto o professor quanto o aluno podem acessar a plataforma diretamente do Moodle com um único clique.
\end{itemize}

\subsection{Desvantagens}
\begin{itemize}
    \item \textbf{Desenvolvimento externo e curva de aprendizado:} O desenvolvimento dos códigos e a criação das listas de exercícios ainda ocorre fora do Moodle, no Beecrowd, o que significa que o professor precisa aprender a usar o Beecrowd Academic, embora o processo seja relativamente simples. Isso pode exigir um tempo adicional de adaptação, especialmente para aqueles que não estão familiarizados com a plataforma.
\end{itemize}

\section{Sistema Especialista para Resolução de Dúvidas no Beecrowd}

\subsection{Vantagens}
\begin{itemize}
    \item \textbf{Atendimento automatizado:} O sistema especialista pode fornecer respostas rápidas e precisas para questões frequentes ou comuns, permitindo que os alunos resolvam suas dúvidas sem a necessidade de intervenção humana constante.
    \item \textbf{Apoio no aprendizado:} O sistema pode ser configurado para fornecer explicações detalhadas sobre questões específicas do Beecrowd, ajudando os alunos a entender melhor os problemas e as soluções.
    \item \textbf{Escalabilidade:} Como o sistema é automatizado, ele pode lidar com um grande volume de dúvidas de maneira eficiente.
    \item \textbf{Redução da sobrecarga para os professores:} Ao automatizar o atendimento às dúvidas mais comuns, o sistema especialista permite que os professores se concentrem em questões mais complexas e em atividades pedagógicas, sem se sobrecarregar com um grande volume de perguntas repetitivas.
    \item \textbf{Auxílio para professores não familiarizados com o Beecrowd:} Para professores que ainda não estão familiarizados com as questões do Beecrowd, o sistema especialista oferece uma ferramenta valiosa para auxiliá-los a responder dúvidas dos alunos de maneira eficiente, sem a necessidade de um profundo conhecimento prévio sobre a plataforma ou as questões.
\end{itemize}

\subsection{Desvantagens}
\begin{itemize}
	\item \textbf{Limitação nas respostas para questões complexas:} O sistema especialista é eficiente para lidar com questões simples ou frequentes, mas pode encontrar dificuldades ao lidar com questões mais complexas ou específicas. Para adicionar suporte a novas questões, o professor precisa criar uma matriz de decisões detalhada, o que pode tornar o processo demorado e propenso a erros, especialmente quando se trata de questões que exigem uma análise mais aprofundada. Sem a devida cautela, o sistema pode acabar gerando respostas imprecisas ou inadequadas, comprometendo a qualidade das interações e a efetividade do suporte oferecido.
    \item \textbf{Dependência da qualidade da base de conhecimento:} O desempenho do sistema especialista depende diretamente da qualidade das regras e conhecimentos que foram programados nele. Se a base de dados for limitada ou desatualizada, o sistema pode fornecer respostas inadequadas.
    \item \textbf{Falta de interação humana:} Em casos onde o aluno necessita de uma explicação mais personalizada ou de orientação emocional, o sistema especialista pode não ser suficiente, criando uma experiência impessoal que pode não atender completamente às necessidades do aluno.
\end{itemize}

\section{Avaliação Prática}

A avaliação prática foi conduzida por meio da utilização do sistema especialista pelos alunos da disciplina \textit{Programação Orientada a Objetos I} da UFSC, durante o segundo semestre de 2024. Para isso, foram realizados os seguintes passos:

\begin{enumerate}
    \item \textbf{Aplicação do sistema:} Os alunos foram incentivados a usar o sistema especialista para resolver dúvidas relacionadas às questões do Beecrowd. Durante esse período, o uso do sistema foi monitorado para avaliar sua eficácia em termos de resolução de dúvidas.
    \item \textbf{Análise de resultados:} Foram avaliados o número de dúvidas resolvidas, as dúvidas não esclarecidas e os \textit{feedbacks} fornecidos pelos alunos.
    \item \textbf{Melhorias das questões:} Com base na análise dos resultados, dúvidas não resolvidas pelo Bee Expert foram incluídas no sistema para futuras turmas.
\end{enumerate}

É importante destacar que, até o momento, o Expert Bee tinha apenas dicas de resoluções para essas questões da lista de exercícios fornecida, e não dicas para dúvidas reais coletadas de alunos. Essas dicas de resoluções tinham sido disponibilizadas pelo professor durante a primeira coleta de dados, descrita na seção \ref{sec:coleta-dados}, para que fosse possível testar a eficácia da ferramenta na hora em que as dúvidas dos alunos surgiriam. 

Assim, um dos objetivos da avaliação prática também era identificar dúvidas não tratadas pelo Expert Bee, para que elas pudessem ser adicionadas à ferramenta.

\subsection{Resultados}  

Na Avaliação Prática realizada em 18 de novembro de 2024, foi disponibilizada aos alunos uma lista com as seguintes questões do Beecrowd: 1261, 1281, 1430, 1449, 1483, 1763, 1911, 1953, 2091, 2482, 2492, 2654, 2949 e 2987, para que começassem a resolver durante a aula. 12 alunos estiveram presente na sala de aula e, sempre que um aluno apresentava uma dúvida, ele utilizava primeiro o Expert Bee para tentar resolvê-la, e, caso não conseguisse, recorria ao professor. 

As dúvidas relatadas pelos alunos foram as seguintes:  

\begin{itemize}  
    \item \textbf{Questão 1911:} 4 alunos apresentaram dúvidas, e todos conseguiram resolvê-las apenas com o auxílio do Bee Expert.  
    \item \textbf{Questão 2654:} 2 alunos tiveram dúvidas, todas solucionadas pelo Bee Expert.  
    \item \textbf{Questão 1430:} 1 aluno apresentou dúvida e conseguiu resolvê-la com o Bee Expert.  
    \item \textbf{Questão 2949:} 1 aluno apresentou dúvida, solucionada pelo Bee Expert.  
    \item \textbf{Questão 1483:} 1 aluno teve dúvida, também solucionada pelo Bee Expert.  
    \item \textbf{Questão 1261:} 2 alunos apresentaram dúvidas específicas que não puderam ser resolvidas, pois o Bee Expert, à época, fornecia apenas uma ideia geral de resolução.  
\end{itemize}  

No total, de 11 dúvidas relatadas, o Bee Expert conseguiu resolver 9, resultando em uma taxa de sucesso de aproximadamente 81\%.  

\subsection{Pontos de Melhoria}  

\begin{itemize}  
    \item \textbf{Questão 1261:} As dúvidas específicas dos alunos foram incorporadas ao sistema, enriquecendo o conteúdo da questão para turmas futuras.  
    \item \textbf{Questões 1911 e 2654:} Apesar do sucesso na resolução, foi identificado um ponto de melhoria: o Bee Expert apresentava diretamente a solução completa após a primeira interação. No entanto, os alunos demonstraram interesse em orientações iniciais sobre como resolver as questões utilizando dicionários em Python. Esse ajuste foi implementado, permitindo ao usuário solicitar essas informações antes da solução.  
    \item \textbf{Questão 1430:} A dúvida do aluno surgiu por falta de entendimento do método \texttt{split()} do Python. Como melhoria, o arquivo dessa questão no Bee Expert agora inclui uma explicação detalhada sobre o uso do método.  
\end{itemize}  

\end{otherlanguage*}

