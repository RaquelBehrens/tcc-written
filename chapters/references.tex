
% The \phantomsection command is needed to create a link to a place in the document that is not a
% figure, equation, table, section, subsection, chapter, etc.
% https://tex.stackexchange.com/questions/44088/when-do-i-need-to-invoke-phantomsection
\phantomsection

% Multiple-language document - babel - selectlanguage vs begin/end{otherlanguage}
% https://tex.stackexchange.com/questions/36526/multiple-language-document-babel-selectlanguage-vs-begin-endotherlanguage
\begin{otherlanguage*}{brazil}

    \chapter{REFERENCES}

LIMA, José Maria Maciel. Plataforma Moodle: a educação por mediação tecnológica. Revista Científica Multidisciplinar Núcleo do Conhecimento, Ano 06, Ed. 01, v. 7, p. 17-37, 28 jan. 2021. Revista Científica Multidisciplinar Núcleo Do Conhecimento. http://dx.doi.org/10.32749/nucleodoconhecimento.com.br/educacao/plataforma-moodle. ISSN: 2448-0959. Disponível em: https://www.nucleodoconhecimento.com.br/educacao/plataforma-moodle. Acesso em: 19 jun. 2023.

FRANÇA, Allyson Bonetti; SOARES, José Marques. Sistema de apoio a atividades de laboratório de programação via Moodle com suporte ao balanceamento de carga. In: XXII SBIE - XVII WIE, 22., 2011, Aracaju. Anais do XXII SBIE - XVII WIE. Fortaleza: Sbie, 2011. p. 710-719. Disponível em: https://www.researchgate.net/profile/Jose-Soares-14/publication/268337343_Sistema_de_apoio_a_atividades_de_laboratorio_de_programacao_via_Moodle_com_suporte_ao_balanceamento_de_carga/links/564309dd08aeacfd8938a73a/Sistema-de-apoio-a-atividades-de-laboratorio-de-programacao-via-Moodle-com-suporte-ao-balanceamento-de-carga.pdf. Acesso em: 19 jun. 2023.

RODRÍGUEZ-DEL-PINO, Juan Carlos. Moodle plugins directory: Virtual Programming Lab. 2023. Disponível em: https://moodle.org/plugins/mod_vpl. Acesso em: 19 jun. 2023.

FREITAS, Larissa Mage de. Análise de Usabilidade do Módulo Laboratório Virtual de Programação do Moodle. 2016. 146 f. TCC (Graduação) - Curso de Tecnologias da Informação e Comunicação, Universidade Federal de Santa Catarina, Araranguá, 2016. Disponível em: https://repositorio.ufsc.br/handle/123456789/165178. Acesso em: 19 jun. 2023.

CRUZ, Allan Kássio Beckman Soares da; SOARES NETO, Carlos de Salles; CRUZ, Pamela Torres Maia Beckman da; TEIXEIRA, Mário Antonio Meireles. Utilização da Plataforma Beecrowd de Maratona de Programação como Estratégia para o Ensino de Algoritmos. Anais Estendidos do XXI Simpósio Brasileiro de Jogos e Entretenimento Digital (Sbgames Estendido 2022), Natal/RN, p. 754-764, 24 out. 2022. Sociedade Brasileira de Computação. http://dx.doi.org/10.5753/sbgames_estendido.2022.225898. Disponível em: https://doi.org/10.5753/sbgames_estendido.2022.225898. Acesso em: 20 jun. 2023.

BEZ, Jean Luca; TONIN, Neilor A.. URI Online Judge: A New Classroom Tool for Interactive Learning. The Steering Committee Of The World Congress In Computer Science, Computer Engineering And Applied Computing (Worldcomp), Athens, p. 1-5, 2012. Disponível em: https://search.proquest.com/openview/33325d7946128d0f1096abbe8b0b3664/1?pq-origsite=gscholar&cbl=1976352. Acesso em: 20 jun. 2023.

BEZ, Jean Luca; TONIN, Neilor A.. URI Online Judge e a Internacionalização da Universidade. Revista Eletrônica de Extensão da Uri, Erechim, v. 10, n. 18, p. 237-249, maio 2014. Disponível em: https://docplayer.com.br/11099240-Uri-online-judge-e-a-internacionalizacao-da-universidade-uri-online-judge-and-the-university-internationalization.html. Acesso em: 20 jun. 2023.

BERSSANETTE, João Henrique; FRANCISCO, Antonio Carlos de. Uma Proposta de Ensino de Programação de Computadores com base na PBL utilizando o portal URI Online Judge. In: II SIMPÓSIO IBERO-AMERICANO DE TECNOLOGIAS EDUCACIONAIS, 2., 2018, Araranguá. Araranguá: Sited, 2018. p. 348-354. Disponível em: https://www.academia.edu/download/83230343/428-25-1253-1-10-20180622.pdf. Acesso em: 20 nov. 2023.

FERREIRA, Fabiana Zaffalon; SOUZA, Ricardo Lemos de; VARGAS, André Prisco; TEIXEIRA, Davi de Lemos; SANTOS, Michel Neves dos; PAES, Wanderson de Oliveira; SANTOS, Rafael Augusto Penna dos; TONIN, Neilor; EVALD, Paulo Jefferson Dias de Oliveira; BOTELHO, Silvia Silva da Costa. Model for evaluation of multiple abilities programming problems in online massive environments. Journal Of The Brazilian Computer Society, [S.L.], v. 28, n. 1, p. 104-117, 3 jan. 2023. Sociedade Brasileira de Computacao - SB. http://dx.doi.org/10.5753/jbcs.2022.2744. Disponível em: https://sol.sbc.org.br/journals/index.php/jbcs/article/view/2744. Acesso em: 20 jun. 2023.

FERREIRA, Rafael Makaha Gomes. BROMS : Brazilian Online Marathon Scoreboard. 2022. 60 f. TCC (Graduação) - Curso de Engenharia de Software, Faculdade Unb Gama, Universidade de Brasília, Brasília, 2021. Disponível em: https://bdm.unb.br/handle/10483/30740. Acesso em: 06 nov. 2023.

LIMA, Gustavo Marques. Gamma Online Judge : projeto de criação de uma plataforma para o armazenamento das questões das maratonas UnB de programação. 2023. 77 f. TCC (Graduação) - Curso de Bacharelado em Engenharia de Software, Universidade de Brasília, Faculdade Unb Gama, Brasília, 2022. Disponível em: https://bdm.unb.br/handle/10483/34516. Acesso em: 06 nov. 2023.

FRANCISCO, Rodrigo; PEREIRA JÚNIOR, Cleon; AMBRÓSIO, Ana Paula. Juiz Online no ensino de Programação Introdutória - Uma Revisao Sistemática da Literatura. Anais do XXVII Simpósio Brasileiro de Informática na Educação (Sbie 2016), [S.L.], p. 11-20, 7 nov. 2016. Sociedade Brasileira de Computação - SBC. http://dx.doi.org/10.5753/cbie.sbie.2016.11. Disponível em: https://www.researchgate.net/profile/Cleon-Pereira-Junior/publication/309778045_Juiz_Online_no_Ensino_de_Programacao_Introdutoria_-_Uma_Revisao_Sistematica_da_Literatura/links/582318d608aeb45b58893e6d/Juiz-Online-no-Ensino-de-Programacao-Introdutoria-Uma-Revisao-Sistematica-da-Literatura.pdf. Acesso em: 08 nov. 2023.

GALVÃO, Leandro; FERNANDES, David; GADELHA, Bruno. Juiz online como ferramenta de apoio a uma metodologia de ensino híbrido em programação. Anais do XXVII Simpósio Brasileiro de Informática na Educação (Sbie 2016), [S.L.], p. 140-149, 7 nov. 2016. Sociedade Brasileira de Computação - SBC. http://dx.doi.org/10.5753/cbie.sbie.2016.140. Disponível em: https://www.researchgate.net/profile/Leandro-Carvalho-3/publication/309892315_Juiz_online_como_ferramenta_de_apoio_a_uma_metodologia_de_ensino_hibrido_em_programacao/links/5da882a792851caa1babdcc5/Juiz-online-como-ferramenta-de-apoio-a-uma-metodologia-de-ensino-hibrido-em-programacao.pdf. Acesso em: 08 nov. 2023

WASIK, Szymon; ANTCZAK, Maciej; BADURA, Jan; LASKOWSKI, Artur; STERNAL, Tomasz. A Survey on Online Judge Systems and Their Applications. Acm Computing Surveys, [S.L.], v. 51, n. 1, p. 1-34, 4 jan. 2018. Association for Computing Machinery (ACM). http://dx.doi.org/10.1145/3143560. Disponível em: https://dl.acm.org/doi/abs/10.1145/3143560. Acesso em: 08 nov. 2023.

SANTOS, Joanna C. S.; RIBEIRO, Admilson R. L.. JOnline: proposta preliminar de um juiz online didático para o ensino de programação. Anais do Simpósio Brasileiro de Informática na Educação (Sbie 2011), Aracaju, v. 5, n. 8, p. 964-967, nov. 2011. Disponível em: http://milanesa.ime.usp.br/rbie/index.php/sbie/article/view/1863. Acesso em: 08 nov. 2023.

RIBEIRO, Ralph Breno; FERNANDES, David; CARVALHO, Leandro Silva Galvão de; OLIVEIRA, Elaine. Gamificação de um Sistema de Juiz Online para Motivar Alunos em Disciplina de Programação Introdutória. Anais do XXIX Simpósio Brasileiro de Informática na Educação (Sbie 2018), [S.L.], p. 805-814, 28 out. 2018. Brazilian Computer Society (Sociedade Brasileira de Computação - SBC). http://dx.doi.org/10.5753/cbie.sbie.2018.805. Disponível em: http://milanesa.ime.usp.br/rbie/index.php/sbie/article/view/8040. Acesso em: 08 nov. 2023.

GALVÃO, Leandro; FERNANDES, David; GADELHA, Bruno. Juiz online como ferramenta de apoio a uma metodologia de ensino híbrido em programação. Anais do XXVII Simpósio Brasileiro de Informática na Educação (Sbie 2016), [S.L.], v. 27, n. 1, p. 140-149, 7 nov. 2016. Sociedade Brasileira de Computação - SBC. http://dx.doi.org/10.5753/cbie.sbie.2016.140. Acesso em: 08 nov. 2023.

SALES, André Barros de; COSTA JUNIOR, Edson; SALES, Márcia Barros de. Utilização de Problemas da Maratona de Competição de Programação e Juízes Eletrônicos como Estratégia de Ensino em um Curso de Graduação em Engenharia de Software. Anais do XXVII Simpósio Brasileiro de Informática na Educação (Sbie 2016), [S.L.], v. 5, n. 8, p. 210-219, 7 nov. 2016. Sociedade Brasileira de Computação - SBC. http://dx.doi.org/10.5753/cbie.sbie.2016.210. Disponível em: http://milanesa.ime.usp.br/rbie/index.php/sbie/article/view/6701. Acesso em: 08 nov. 2023.

CAMPOS, Cassio P. de; FERREIRA, Carlos E.. BOCA: um sistema de apoio a competições de programação. Sociedade Brasileira de Computação, São Paulo, 2004. Disponível em: https://research.tue.nl/en/publications/boca-um-sistema-de-apoio-a-competi%C3%A7%C3%B5es-de-programa%C3%A7%C3%A3o. Acesso em: 08 nov. 2023.

BEECROWD. Beecrowd,  2021.  Página  inicial.  Disponível  em:  https://www.beecrowd.com.br/judge/pt.  Acesso em: 09 set. 2023.

BEECROWD ACADEMIC. Beecrowd Academic,  2021.  Página  inicial.  Disponível  em:  https://www.beecrowd.com.br/academic/pt.  Acesso em: 09 set. 2023.

PIEKARSKI, Ana Elisa Tozetto; MIAZAKI, Mauro; ROCHA JUNIOR, Alexandro Luis da; MILITÃO, Eric Patrick; SILVA, João Vitor Pieczarka da. PROGRAMAÇÃO COMPETITIVA EM UM PROJETO DE EXTENSÃO PARA O ENSINO TÉCNICO EM INFORMÁTICA. Revista Conexão Uepg, Ponta Grossa, v. 19, n. 1, p. 1-15, 2023. Universidade Estadual de Ponta Grossa (UEPG). http://dx.doi.org/10.5212/rev.conexao.v.19.21239.018. Disponível em: https://revistas.uepg.br/index.php/conexao/article/view/21239. Acesso em: 9 nov. 2023.

GALASSO, Rafael Hernandez; MOREIRA, Benjamin Grando. Integração do ambiente BOCA com o ambiente Moodle para avaliação automática de algoritmos. In: COMPUTER ON THE BEACH, 2014, Florianópolis. Florianópolis: Computer On The Beach, 2014. p. 22-31. Disponível em: https://periodicos.univali.br/index.php/acotb/article/view/5292. Acesso em: 9 nov. 2023.
MOODLE. Moodle Documentation. 2023. Disponível em: https://docs.moodle.org/. Acesso em: 10 nov. 2023.

MOODLE. Moodle Developer Resource centre. 2023. Disponível em: https://moodledev.io/. Acesso em: 10 nov. 2023.

VPL. Virtual Programming Lab for Moodle (VPL). 2021. Disponível em: https://vpl.dis.ulpgc.es/. Acesso em: 13 nov. 2023.

CODERUNNER. CodeRunner. Disponível em: https://coderunner.org.nz/. Acesso em: 13 nov. 2023.

MASSÉ, Mark. REST API Design Rulebook. 1005 Gravenstein Highway North, Sebastopol: O’reilly Media, Inc., 2011. Disponível em: https://books.google.com.br/books?hl=pt-PT&lr=&id=eABpzyTcJNIC&oi=fnd&pg=PR3&dq=REST+API+Design+Rulebook&ots=vBOvZ_jdMD&sig=tvXanZ39ViurHJ7fWrRIovBwlgI. Acesso em: 13 nov. 2023.

RICHARDSON, Leonard; AMUNDSEN, Mike; RUBY, Sam. RESTful Web APIs. 1005 Gravenstein Highway North, Sebastopol: O’reilly Media, Inc., 2013. Disponível em: https://books.google.com.br/books?hl=pt-PT&lr=&id=wWnGAAAAQBAJ&oi=fnd&pg=PR2&dq=restful+web+apis&ots=Fi9itG-75e&sig=q64KUvXJzi4zFGQhnJnIzxjL2LM. Acesso em: 13 nov. 2023.

VICKERS, Stephen P; BOOTH, Simon. Learning Tools Interoperability (LTI): A Best Practice Guide. Creating Environments For Learning Through Instigating A Community Of Developers: A JISC-funded Project, ago. 2014. Disponível em: https://ltiapps.net/guide/LTI_Best_Practice.pdf. Acesso em: 14 nov. 2023.

VERDAGUER, Júlia. O que é LTI e como ele pode melhorar seu ecossistema de aprendizagem. 2021. Moodle. Disponível em: https://moodle.com/pt-br/news/o-que-e-e-como-pode-melhorar-seu-ecossistema-de-aprendizagem/. Acesso em: 14 nov. 2023.

LOBB, Richard; HARLOW, Jenny. Coderunner: a tool for assessing computer programming skills. Acm Inroads, [S.L.], v. 7, n. 1, p. 47-51, 12 fev. 2016. Association for Computing Machinery (ACM). http://dx.doi.org/10.1145/2810041. Disponível em: https://dl.acm.org/doi/abs/10.1145/2810041?casa_token=cV5554XMVswAAAAA:8SVdEhKBneuW43kadkaSv8ELNdN41F376mTCSo7a98yPV9CJFAp-8vADXTAbD9nURnOLmzne-W_f. Acesso em: 16 nov. 2023.
RODRÍGUEZ-DEL-PINO, Juan Carlos; ROYO, Enrique Rubio; FIGUEROA, Zenón Hernández. A virtual programming lab for Moodle with automatic assessment and anti-plagiarism features. In: THE 2012 INTERNATIONAL CONFERENCE ON E-LEARNING, E-BUSINESS, ENTERPRISE INFORMATION SYSTEMS, & E-GOVERNMENT. ISBN: 1-60132-209-7, 7., 2012, Hong Kong. Ata de Congresso. Las Palmas de Gran Canaria, Espanha: IU de Cibernética, Empresa y Sociedad (IUCES), 2012. Disponível em: http://hdl.handle.net/10553/9773. Acesso em: 16 nov. 2023.


\end{otherlanguage*}

