
% The \phantomsection command is needed to create a link to a place in the document that is not a
% figure, equation, table, section, subsection, chapter, etc.
% https://tex.stackexchange.com/questions/44088/when-do-i-need-to-invoke-phantomsection
\phantomsection

% ---
\chapter{\lang{Final Remarks}{Considerações Finais}}

Em conclusão, este trabalho abordou a relevância da integração de ferramentas tecnológicas no ensino de programação, com foco especial no uso de plataformas como Beecrowd e Moodle. A análise dos juízes online e outras plataformas educacionais, como o VPL, evidenciou desafios na usabilidade e na integração, mas também destacou o Beecrowd como uma solução eficaz para o ensino de algoritmos e programação, com recursos que incentivam o aprendizado ativo, como gamificação, \textit{feedback} em tempo real e uma gama diversificada de questões.

Este estudo aprofundou o impacto dos juízes online no ensino de programação, enfatizando a relevância da integração entre plataformas como o Beecrowd e o Moodle. Foi realizada uma análise detalhada do Moodle, destacando suas funcionalidades e como ele pode ser integrado de forma eficiente ao Beecrowd para potencializar o processo de aprendizagem. A proposta de integração dessas duas plataformas, por meio de um plugin LTI (\textit{Learning Tools Interoperability}) desenvolvido pelo Beecrowd, visa simplificar o ensino e tornar as atividades de programação mais acessíveis e eficazes para os alunos.

A integração do Beecrowd ao Moodle, por meio da tecnologia LTI, facilita o acesso de professores e alunos a uma vasta gama de questões, otimizando o uso da plataforma no ambiente acadêmico. Além disso, oferece uma interface amigável para a criação de listas de exercícios pelos professores e a resolução de problemas de programação pelos alunos. Com o objetivo de apoiar essa integração, foi desenvolvido um manual detalhado sobre como integrar o Moodle ao Beecrowd via LTI, bem como um guia para orientar os professores no uso da plataforma.

Para estimular o uso do Beecrowd no ambiente acadêmico, também foi criada uma ferramenta com um sistema especialista que auxilia os alunos ao esclarecer dúvidas recorrentes nas questões do Beecrowd. Esse sistema, testado em sala de aula, não só oferece suporte aos estudantes, mas também facilita a adaptação dos professores, promovendo um ambiente de aprendizado mais fluido e eficiente para todos os envolvidos.

\section{Trabalhos Futuros}

A seção a seguir apresenta algumas sugestões de melhorias e direções futuras para a aplicação de sistema especialista desenvolvida, a Expert Bee. 

\begin{itemize}
    \item  \textbf{Adição de dicas de novas questões pela interface web:} Uma melhoria importante seria adaptar o sistema para permitir que os professores possam adicionar dicas para novas questões diretamente pela interface da aplicação web, sem a necessidade de modificar o código-fonte. Essa melhoria tornaria o sistema mais acessível e eficiente, permitindo que os educadores personalizem o conteúdo de maneira prática e intuitiva, sem a dependência de habilidades técnicas.
    \item  \textbf{Sistema autoalimentado (aprendizado com as interações dos alunos):} Uma possível evolução seria adaptar o sistema para se autoalimentar com base nas interações dos alunos. O sistema poderia aprender com as dúvidas mais frequentes e se ajustar automaticamente, oferecendo respostas e dicas mais precisas com o tempo. Essa melhoria contribuiria para uma experiência de aprendizagem mais personalizada, permitindo que os alunos superem suas dificuldades de forma mais eficiente.
    \item  \textbf{Integração com ferramentas de processamento de linguagem natural (ChatGPT):} Uma adaptação interessante seria integrar o sistema especialista com ferramentas de processamento de linguagem natural, como o ChatGPT. Isso permitiria uma interação mais fluida e natural entre os alunos e o sistema, além de melhorar a precisão, contextualização e dinamismo das respostas. Com essa integração, seria possível oferecer um nível maior de personalização nas interações, aprimorando a qualidade do suporte oferecido.
    \item  \textbf{Adição de dicas de novas questões de outras plataformas:} O Expert Bee também pode ser utilizado para fornecer dicas de questões de outras plataformas além do Beecrowd. Um possível aprimoramento seria adaptar o Expert Bee para identificar de qual plataforma as questões estão vindo e oferecer dicas específicas para essas questões. Dessa forma, seria possível ampliar a utilidade da ferramenta para diferentes plataformas de ensino, tornando-a mais flexível e adaptada a diferentes contextos de aprendizagem.
\end{itemize}


\phantomsection


