
% The \phantomsection command is needed to create a link to a place in the document that is not a
% figure, equation, table, section, subsection, chapter, etc.
% https://tex.stackexchange.com/questions/44088/when-do-i-need-to-invoke-phantomsection
\phantomsection

% ---
\chapter{\lang{Final Remarks}{Considerações Finais}}

Em conclusão, o presente trabalho abordou a relevância da integração de ferramentas tecnológicas, especialmente no contexto educacional da área de tecnologia da informação. A análise de plataformas como o VPL no ambiente Moodle revelou desafios relacionados à interface gráfica e usabilidade, bem como questões de balanceamento de carga. No entanto, ao comparar alternativas, destaca-se a plataforma Beecrowd como uma solução abrangente e eficaz para o ensino de algoritmos e programação.

A Beecrowd se destaca não apenas por oferecer questões em português brasileiro, aumentando sua acessibilidade, mas também por fornecer uma gama completa de recursos, como gamificação, feedback em tempo real, classificação de problemas e níveis de dificuldade. A plataforma não só atende às necessidades dos alunos, incentivando a resolução de problemas e o desenvolvimento de algoritmos, mas também facilita o trabalho dos professores, oferecendo um módulo acadêmico completo para o acompanhamento do desempenho dos alunos.

Diante da demanda crescente e do reconhecimento internacional do Beecrowd, é evidente que essa plataforma representa uma opção sólida para aprimorar o ensino de programação. Sua aceitação positiva pela comunidade acadêmica, aliada aos seus recursos avançados, sugere que o Beecrowd pode desempenhar um papel significativo na promoção do aprendizado ativo e personalizado. Portanto, incentivar e integrar essa ferramenta inovadora nos ambientes acadêmicos pode contribuir significativamente para a qualidade do ensino de programação e solução de problemas nas salas de aula.

Assim, a proposta de integração do Beecrowd ao Moodle por meio de um plugin visa facilitar o uso dessa plataforma pelos educandos, proporcionando maior controle aos docentes sobre as atividades. Os objetivos específicos incluem a transferência automática de notas, o cadastro automático de alunos, a visualização de submissões e feedbacks, entre outros recursos. A proposta busca unificar o processo educacional, permitindo que professores e alunos acessem e interajam com o Beecrowd diretamente no ambiente Moodle. O plugin visa aprimorar a experiência de ensino, proporcionando praticidade, personalização e eficiência na gestão de atividades de programação e algoritmos.

Portanto, durante o trabalho, estabeleceu as bases teóricas necessárias para o entendimento do contexto em que a proposta se insere, proporcionando uma visão abrangente dos elementos que compõem o ambiente educacional digital e os desafios a serem abordados na busca por uma solução integrada e eficiente.

Explorou-se o impacto dos juízes online na formação dos estudantes, destacando a importância dessa tecnologia no contexto educacional. Em seguida, foram apresentadas as características da plataforma Beecrowd, ressaltando sua relevância no ensino de programação, e do Moodle, ambiente virtual de aprendizagem amplamente utilizado.

Quanto ao Beecrowd, é essencial salientar os atributos e funcionalidades presentes na plataforma, proporcionando uma visão abrangente de suas capacidades. Por outro lado, no que diz respeito ao Moodle, vale ressaltar a variedade de módulos e funcionalidades disponíveis para oferecer suporte aos usuários, incluindo tarefas, questionários, e outros recursos. Essa análise permite uma compreensão mais aprofundada sobre qual módulo seria mais adequado para a criação de atividades integradas com o Beecrowd.

A introdução do conceito de plugin foi essencial para contextualizar a proposta de integração do Beecrowd ao Moodle, apontando para a criação de uma solução que visa facilitar e aprimorar a experiência de docentes e discentes, e mostrando características necessárias no desenvolvimento de um plugin do Moodle. O contraste com o BOCA Online Contest Administrator evidenciou as limitações deste em sala de aula, destacando a necessidade de soluções mais adequadas às demandas educacionais contemporâneas.

Além disso, a breve introdução aos conceitos de API e LTI proporcionou uma visão inicial sobre as ferramentas que serão utilizadas no desenvolvimento da solução proposta. Esses elementos formam o alicerce para a compreensão do caminho a ser percorrido na implementação do plugin, integrando o Beecrowd ao Moodle de maneira eficaz.

Após isso, foi apresentado um artigo que delineia um modelo de plano de ensino fundamentado na plataforma Beecrowd. As justificativas e os aspectos positivos destacados reforçam a importância de estratégias inovadoras para aprimorar o processo de ensino e aprendizagem, especialmente na área de programação e algoritmos.

Também foi realizada uma análise de três artigos que abordam a integração de diferentes ambientes de desenvolvimento com o Moodle. Os estudos sobre o BOCA, o CodeRunner e o VPL oferecem insights valiosos sobre abordagens anteriores na busca por soluções eficientes no contexto educacional. Essas análises proporcionam um panorama das experiências anteriores, permitindo identificar desafios e oportunidades na integração de ferramentas tecnológicas com o Moodle.

A compreensão desses estudos serve como base para a proposta apresentada neste trabalho, que visa integrar o Beecrowd ao Moodle por meio de um plugin. A busca por uma solução que otimize o uso dessas plataformas, atendendo às necessidades específicas de cursos de programação, reflete o compromisso com a inovação e a eficácia no ensino.

No decorrer do capítulo de Desenvolvimento, realizou-se uma análise detalhada dos aspectos fundamentais para a integração proposta entre o Beecrowd e o Moodle. A apresentação de diagramas UML desempenhou um papel crucial na definição de uma arquitetura abrangente para a integração, proporcionando uma visão clara e estruturada do processo. Essa abordagem, orientada pela análise contínua da plataforma Beecrowd, permitiu o desenvolvimento de uma proposta alinhada com as necessidades específicas de cursos de programação e algoritmos. Ao unir a compreensão detalhada da plataforma Beecrowd, a apresentação de uma arquitetura robusta e o estudo contínuo da integração proposta, este capítulo estabelece as bases necessárias para a implementação eficaz da solução.

Por fim, foi apresentado um panorama abrangente das atividades já concluídas no âmbito deste Trabalho de Conclusão de Curso (TCC), destacando os avanços alcançados até o momento, e mostrou-se o que falta a ser concluído: a implementação prática da integração proposta, traduzindo os conceitos e planos discutidos neste capítulo em uma solução funcional. Será necessário desenvolver o plugin para o Moodle e realizar testes para garantir a eficácia e a usabilidade da integração. Além disso, a análise contínua da plataforma Beecrowd permanecerá uma parte fundamental do processo, permitindo ajustes e refinamentos conforme necessário.

Uma tabela que distribuirá as tarefas ao longo do tempo foi elaborada, proporcionando um guia claro para o cronograma de implementação, permitindo uma gestão eficiente do tempo e recursos disponíveis. A tabela servirá como um instrumento prático para acompanhar o progresso, garantindo que cada etapa seja abordada de maneira sistemática e coerente. Com a continuidade desse planejamento estruturado, espera-se alcançar os objetivos propostos e contribuir significativamente para o avanço da proposta de integração Beecrowd-Moodle.

\phantomsection


